\documentclass[italian]{article}
\usepackage[T1]{fontenc}
\usepackage{graphicx}
\usepackage{mathtools}
\usepackage{amssymb}
\usepackage{amsthm}
\usepackage{xcolor}
\usepackage{nameref}
\usepackage{babel}
\usepackage[hidelinks]{hyperref}
\graphicspath{img/}
\title{Ingegneria dei requisiti}
\author{Riccardo Torre}
\begin{document}
	\maketitle
	\tableofcontents
	\section{Introduzione}
	Una soluzione software corretta viene applicata per risolvere alcuni dei problemi del mondo reale. È importante comprendere il \textbf{contesto} e i \textbf{problemi} del \textbf{mondo reale}.
	Viene riportato l'esempio del freno a mano di una macchina a guida autonoma.
	Il \textbf{problema} è rappresentato dal fatto che in certe situazioni potrebbe non essere conveniente togliere il freno a mano. Il \textbf{contesto} è rappresentato da molteplici situazioni, quali ad esempio: la macchina è in movimento, la macchina sta frenando, le intenzioni dell'autista, le regole di sicurezza, e così via\dots

	\paragraph{Terminologia: mondo e macchina.} Il \textbf{mondo} è usato come riferimento a tutte le problematiche del mondo reale. Può contenere \textbf{componenti umane}, come le organizzazioni, gli staff, gli operatori\dots, ma anche delle \textbf{componenti fisiche} come i dispositivi, il software legacy, i sensori, gli attuatori, madre natura (le componenti fisiche vanno modellate correttamente e fanno parte di madre natura)\dots
	Quando ci si utilizza il termine \textbf{macchina} si fa riferimento al mondo dei computer che sono dei contenitori software e hardware.

	L'ingegneria dei requisiti si occupa:
	\begin{itemize}
		\item dell'\textbf{effetto} che il software ha sul mondo reale;
		\item delle \textbf{proprietà} del mondo reale che il software deve tenere in considerazione.
	\end{itemize}
	\begin{figure}[th]
		\centering
		\includegraphics[width=\linewidth]{img/word-machine-sets}
		\caption{modellazione dei problemi che sono contenuti nel mondo reale e delle soluzioni che sono contenute nella macchina.}
		\label{fig:word-machine-sets}
	\end{figure}
	Nella figura \ref{fig:word-machine-sets} vengono rappresentati, il \textbf{mondo reale}, e il \textbf{mondo delle macchine}, ovvero il software.

	Di seguito, viene indicato con $W$ il mondo reale e con $M$ il mondo delle macchine.
	\begin{itemize}
		\item $W \setminus M$: è la parte che rappresenta tutti i fenomeni del mondo reale che non interagiscono con il software. Ad esempio, l'accensione del motore, il rilascio del freno a mano, l'evento che l'utente vuole iniziare a guidare.
		\item $W \cap M$: appartengono tutti quei fenomeni che vengono modellati dal software. Ad esempio, il controllo del freno è spento, oppure il controllore del regime del motore è impostato ad "up".
		\item $M \setminus W$: sono i fenomeni che appartengono esclusivamente al mondo del software. Ad esempio, lo stato del database ad "aggiornato", oppure un codice di errore (errorCode = 013).
	\end{itemize}
	I requisiti descrivono l'informazione che appartiene solo all'insieme $W$, dunque sono legati esclusivamente al funzionamento del mondo reale e non ad aspetti che riguardano solamente il software.

	Esistono due versioni del mondo:
	\begin{enumerate}
		\item \textbf{System-as-is}: è la descrizione del mondo, prima dell'introduzione del sistema software.
		\item \textbf{System-to-be}: è la descrizione del mondo nel quale il sistema opera.
	\end{enumerate}
	\begin{figure}[th]
		\centering
		\includegraphics[width=\linewidth]{img/system-as-is-system-to-be}
		\caption{system-as-is e system-to-be.}
		\label{fig:system-as-is-system-to-be}
	\end{figure}
	In generale l'\textbf{ingegneria dei requisiti} si occupa di coordinare un insieme di attività che ci permettono di esplorare, valutare, documentare, consolidare, revisionare ed adattare gli obbiettivi, le funzionalità, le qualità, i vincoli e le assunzioni di un software.
	I \textbf{requisiti di sistema} fanno parte del system-to-be. I \textbf{requisiti del software} fanno parte del software-to-be e sono le funzionalità che deve fornire per rispondere ai requisiti richiesti dall'utente.
	\begin{figure}[th]
		\centering
		\includegraphics[width=\linewidth]{img/www-dimensions}
		\caption{il chi, che cosa e perché dell'ingegneria dei requisiti.}
		\label{fig:www-dimensions}
	\end{figure}

	Per comprendere appieno il system-to-be (figura \ref{fig:www-dimensions}) bisogna comprendere quali sono gli \emph{obbiettivi} del sistema (\textbf{perché} è necessario un nuovo sistema), bisogna stabilire quali \emph{servizi e funzionalità} sono necessari per soddisfare gli obbiettivi del sistema, descrivendo i vincoli, le assunzioni e i requisiti (\textbf{che cosa}). Deve essere chiaro anche \textbf{chi} ha la \emph{responsabilità} di fornire i servizi (componenti software, dispositivi fisici oppure operatori umani) e quali sono i \emph{ruoli}.

	I registri per descrivere i requisiti possono essere:
	\begin{itemize}
		\item \textbf{descrittivo:} si concentra sulla definizione di proprietà, indipendentemente dalle funzionalità del sistema. Potrebbero dipendere dalle leggi naturali, dai vincoli fisici, e così via. Ad esempio, un requisito descrittivo potrebbe essere:
		\begin{quotation}
			\textit{"Se le porte del treno sono chiuse, allora non sono aperte."}
		\end{quotation}
		Oppure:
		\begin{quotation}
			\textit{"Se l'accelerazione del treno è positiva, allora la sua velocità è non nulla."}
		\end{quotation}
		\item \textbf{prescrittiva:} proprietà desiderabili del sistema che potrebbero dipendere da come il sistema funziona.
		\begin{quotation}
			\textit{Le porte dovrebbero sempre rimanere chiuse quando il treno è in movimento}
		\end{quotation}
	\end{itemize}
\end{document}