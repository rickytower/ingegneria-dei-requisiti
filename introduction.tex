\documentclass[italian]{article}
\usepackage[T1]{fontenc}
\usepackage{graphicx}
\usepackage{mathtools}
\usepackage{amssymb}
\usepackage{amsthm}
\usepackage{listings}
\usepackage{mathtools}
\usepackage{xcolor}
\usepackage{tabularray}
\usepackage{tabularx}
\usepackage{lipsum}
\usepackage{longtable}
\usepackage[normalem]{ulem}
\usepackage{microtype}
\usepackage{babel}
\usepackage[hidelinks]{hyperref}
\newcommand{\Xcancel}[2][black]{{\color{#1}\xcancel{\color{black}#2}}}
\graphicspath{img/}
\DeclareMathSymbol{\mlq}{\mathord}{operators}{``}
\DeclareMathSymbol{\mrq}{\mathord}{operators}{`'}
\title{Ingegneria dei requisiti}
{\renewcommand{\arraystretch}{2}%
\author{Riccardo Torre}
\begin{document}
	\maketitle
	\tableofcontents
	\section{Introduzione}
	Una soluzione software corretta viene applicata per risolvere alcuni dei problemi del mondo reale. È importante comprendere il \textbf{contesto} e i \textbf{problemi} del \textbf{mondo reale}.
	Viene riportato l'esempio del freno a mano di una macchina a guida autonoma.
	Il \textbf{problema} è rappresentato dal fatto che in certe situazioni potrebbe non essere conveniente togliere il freno a mano. Il \textbf{contesto} è rappresentato da molteplici situazioni, quali ad esempio: la macchina è in movimento, la macchina sta frenando, le intenzioni dell'autista, le regole di sicurezza, e così via\dots

	\paragraph{Terminologia: mondo e macchina.} Il \textbf{mondo} è usato come riferimento a tutte le problematiche del mondo reale. Può contenere \textbf{componenti umane}, come le organizzazioni, gli staff, gli operatori\dots, ma anche delle \textbf{componenti fisiche} come i dispositivi, il software legacy, i sensori, gli attuatori, madre natura (le componenti fisiche vanno modellate correttamente e fanno parte di madre natura)\dots
	Quando ci si utilizza il termine \textbf{macchina} si fa riferimento al mondo dei computer che sono dei contenitori software e hardware.

	L'ingegneria dei requisiti si occupa:
	\begin{itemize}
		\item dell'\textbf{effetto} che il software ha sul mondo reale;
		\item delle \textbf{proprietà} del mondo reale che il software deve tenere in considerazione.
	\end{itemize}
	\begin{figure}[th]
		\centering
		\includegraphics[width=0.7\linewidth]{img/word-machine-sets}
		\caption{modellazione dei problemi che sono contenuti nel mondo reale e delle soluzioni che sono contenute nella macchina.}
		\label{fig:word-machine-sets}
	\end{figure}
	Nella figura \ref{fig:word-machine-sets} vengono rappresentati, il \textbf{mondo reale}, e il \textbf{mondo delle macchine}, ovvero il software.

	Di seguito, viene indicato con $W$ il mondo reale e con $M$ il mondo delle macchine.
	\begin{itemize}
		\item $W \setminus M$: è la parte che rappresenta tutti i fenomeni del mondo reale che non interagiscono con il software. Ad esempio, l'accensione del motore, il rilascio del freno a mano, l'evento che l'utente vuole iniziare a guidare.
		\item $W \cap M$: appartengono tutti quei fenomeni che vengono modellati dal software. Ad esempio, il controllo del freno è spento, oppure il controllore del regime del motore è impostato ad "up".
		\item $M \setminus W$: sono i fenomeni che appartengono esclusivamente al mondo del software. Ad esempio, lo stato del database ad "aggiornato", oppure un codice di errore (errorCode = 013).
	\end{itemize}
	I requisiti descrivono l'informazione che appartiene solo all'insieme $W$, dunque sono legati esclusivamente al funzionamento del mondo reale e non ad aspetti che riguardano solamente il software.

	Esistono due versioni del mondo:
	\begin{enumerate}
		\item \textbf{System-as-is}: è la descrizione del mondo, prima dell'introduzione del sistema software.
		\item \textbf{System-to-be}: è la descrizione del mondo nel quale il sistema opera.
	\end{enumerate}
	\begin{figure}[th]
		\centering
		\includegraphics[width=0.7\linewidth]{img/system-as-is-system-to-be}
		\caption{system-as-is e system-to-be.}
		\label{fig:system-as-is-system-to-be}
	\end{figure}
	In generale l'\textbf{ingegneria dei requisiti} si occupa di coordinare un insieme di attività che ci permettono di esplorare, valutare, documentare, consolidare, revisionare ed adattare gli obbiettivi, le funzionalità, le qualità, i vincoli e le assunzioni di un software.
	I \textbf{requisiti di sistema} fanno parte del system-to-be. I \textbf{requisiti del software} fanno parte del software-to-be e sono le funzionalità che deve fornire per rispondere ai requisiti richiesti dall'utente.
	\begin{figure}[th]
		\centering
		\includegraphics[width=0.7\linewidth]{img/www-dimensions}
		\caption{il chi, che cosa e perché dell'ingegneria dei requisiti.}
		\label{fig:www-dimensions}
	\end{figure}

	Per comprendere appieno il system-to-be (figura \ref{fig:www-dimensions}) bisogna comprendere quali sono gli \emph{obbiettivi} del sistema (\textbf{perché} è necessario un nuovo sistema), bisogna stabilire quali \emph{servizi e funzionalità} sono necessari per soddisfare gli obbiettivi del sistema, descrivendo i vincoli, le assunzioni e i requisiti (\textbf{che cosa}). Deve essere chiaro anche \textbf{chi} ha la \emph{responsabilità} di fornire i servizi (componenti software, dispositivi fisici oppure operatori umani) e quali sono i \emph{ruoli}.

	I registri per descrivere i requisiti possono essere:
	\begin{itemize}
		\item \textbf{descrittivo:} si concentra sulla definizione di proprietà, indipendentemente dalle funzionalità del sistema. Potrebbero dipendere dalle leggi naturali, dai vincoli fisici, e così via. Ad esempio, un requisito descrittivo potrebbe essere:
		\begin{quotation}
			\textit{"Se le porte del treno sono chiuse, allora non sono aperte."}
		\end{quotation}
		Oppure:
		\begin{quotation}
			\textit{"Se l'accelerazione del treno è positiva, allora la sua velocità è non nulla."}
		\end{quotation}
		\item \textbf{prescrittiva:} proprietà desiderabili del sistema che potrebbero dipendere da come il sistema funziona.
		\begin{quotation}
			\textit{Le porte dovrebbero sempre rimanere chiuse quando il treno è in movimento}
		\end{quotation}
	\end{itemize}
	\paragraph{\textbf{Nota bene!}}I vincoli prescrittivi possono essere negoziati tra gli ingegneri dei requisiti e i vari stakeholders, possono essere indeboliti o rimpiazzati con alternative, mentre i vincoli descrittivi no.

	I requisiti possono riferirsi a fenomeni dell'\textbf{ambiente} (ad esempio, se il treno si sta muovendo, le porte devono essere chiuse) oppure possono essere condivisi tra il software-to-be e l'ambiente (una componente software potrebbe monitorare un fenomeno ambientale e avvisare un'altra componente software del cambiamento avvenuto. Ad esempio, se la velocità monitorata è pari a zero, allora lo stato delle porte deve essere chiuso).

	Un \textbf{requisito di sistema} ha delle frasi tipicamente \textbf{prescrittive} che si riferiscono ai fenomeni ambientali e che devono essere formulate con vocabolario che deve essere comprensibile a tutti gli stakeholders. Ad esempio:
	\begin{quotation}
		\textit{Se il sistema è in movimento, allora le porte devono essere chiuse}
	\end{quotation}
	ovvero: $\text{TrainMoving} \implies \text{DoorsClosed}$.
	Un \textbf{requisito del software} contiene tipicamente delle frasi \textbf{prescrittive} che si riferiscono a fenomeni condivisi e che devono essere supportate dal software-to-be. Vengono formulate con un vocabolario comprensibile agli sviluppatori software. Un esempio è il seguente.
	\begin{quotation}
		\textit{"Se la velocità misurata è diversa da zero, allora le porte devono essere nello stato chiuso"}
	\end{quotation}
	ossia: $\text{measuredSpeed}\ne 0 \implies doorsState = \text{`closed'}$
	Tipi di frasi che possiamo formulare nei nostri domini possono essere:
	\begin{itemize}
		\item \textbf{Proprietà di dominio:} è una frase descrittiva riguardo un fenomeno del mondo reale che avviene indipendentemente dall'esistenza di un sistema software.
		Ad esempio: $\text{trainAcceleration} > 0 \implies \text{trainSpeed} \ne 0$
		\item \textbf{Assunzioni:} sono delle frasi che devono essere soddisfatte dall'ambiente quando deve essere considerato il software-to-be e formulate in riferimento a fenomeni ambientali. Generalmente sono prescrittive. Ad esempio
		$\text{measuredSpeed} \ne 0 \iff \text{trainSpeed} \ne 0$.
		\item \textbf{Definizioni:} sono frasi che forniscono un significato preciso al sistema. Non hanno un valore di verità. Ad esempio, una definizione potrebbe essere:
		\begin{quotation}
			\textit{measuredSpeed (velocità misurata) è la velocità stimata dal tachimetro del treno}.
		\end{quotation}
	\end{itemize}
	\begin{figure}[th]
		\centering
		\includegraphics[width=0.7\linewidth]{img/modello-4-variabili.drawio}
		\caption{modello 4-variable.}
		\label{fig:modello-4-variabili}
	\end{figure}
	Il modello in figura \ref{fig:modello-4-variabili} è composto da:
	\begin{itemize}
		\item \textbf{dispositivi di input} quali ad esempio sensori;
		\item \textbf{softwareToBe} ovvero il software da sviluppare;
		\item \textbf{ambiente} nel quale il sistema opera;
		\item \textbf{dispositivi di output} ad esempio attuatori.
	\end{itemize}
	Il sistema misura la velocità attraverso i sensori. Il software in esecuzione nel sistema, legge i dati di input, prende delle decisioni, calcola dei risultati che successivamente invia agli attuatori. Gli attuatori rilevano i risultati e attuano delle azioni sul mondo esterno. In questa fase possono modificare delle variabili di controllo del mondo esterno (come quelle delle porte). Il mondo contiene sia le variabili controllate dagli attuatori, sia quelle monitorate dai dispositivi di input. Ad esempio la variabile che contiene l'informazione relativa alla velocità del treno, viene monitorata da uno o più sensori. Quindi la variabile monitorata è la velocità del treno, mentre la variabile controllata è le porte chiuse.
	\begin{itemize}
		\item 	$SysReq \subseteq M \times C$: i requisiti di sistema sono una relazione tra variabili monitorate e le variabili controllate. Le variabili controllate cambiano a seconda delle variabili monitorate;
		\item $SofReq \subseteq I \times O$: i requisiti software sono la relazione tra dati di input ed output;
		\item $SofReq = Map(SysReq, Dom, Asm)$: i requisiti software sono una mappa tra i requisiti di sistema ($SysReq$), le assunzioni ($Asm$) e le proprietà di dominio ($Dom$).
	\end{itemize}
	La mappa $SysReq$ è riportata nella figura \ref{fig:mappa-sysreq} e si può riassumere con la seguente relazione \ref{eqn:mapping-system-req-to-sw-req}:
	\begin{equation}
		\text{SOFREQ, ASM, DOM |= SysReq}
		\label{eqn:mapping-system-req-to-sw-req}
	\end{equation}
	e il cui significato è che i requisiti di sistema vengono soddisfatti se sono sono soddisfatti contemporaneamente i requisiti del software, le assunzioni e le proprietà di dominio.
	\begin{figure}[th]
		\centering
		\begin{eqnarray*}
			\begin{tblr}{cc}
				SoftReq \colon & measuredSpeed \ne 0 \implies doorsState = \mlq closed\mrq\\
				ASM \colon &  measuredSpeed \ne 0 \iff trainSpeed \ne 0 \\
				&doorsState = \mlq closed \mrq \iff DoorsClosed\\
				Dom\colon &  TrainMoving \iff trainSpeed \ne 0\\
				\hline[dashed]
				SysReq\colon&  TrainMoving \implies DoorsClosed
			\end{tblr}
		\end{eqnarray*}
		\caption{I requisiti del software sono una mappa (relazione).}
		\label{fig:mappa-sysreq}
	\end{figure}
	\paragraph{Terminologia: requisiti funzionali e non funzionali.}
	I \textbf{requisiti funzionali} specificano quali sono i servizi e le funzionalità che il software deve fornire. Ad esempio:
	\begin{quotation}
		\textit{"Il software deve controllare l'accelerazione dei treni."}
	\end{quotation}
	è una funzionalità che il software deve provvedere.
	I \textbf{requisiti non funzionali} evidenziano le qualità che le funzionalità devono avere. Esempi tipici sono la sicurezza, l'affidabilità del sistema, le performance in termini di tempo/spazio, l'usabilità, la reattività agli eventi, etc\dots Rappresentano dei vincoli. Un esempio di requisito funzionale è il seguente.
	\begin{quotation}
		\textit{"I comandi di accelerazione devono essere impartiti ogni 3 secondi a ciascun treno."}
	\end{quotation}

	\paragraph{Tassonomia dei requisiti non funzionali.} Viene usata per guidare gli stakeholders nella descrizione dei requisiti non funzionali (che altrimenti, potrebbero non emergere durante l'intervista). In alcuni casi i requisiti non funzionali diventano funzionali (ad esempio, i requisiti di sicurezza usati per specificare le funzionalità di un firewall sono dei requisiti funzionali).

	\subsection{Come evolvono i requisiti}
		I requisiti si \textbf{evolvono nel tempo}.
		\begin{figure}[th]
			\centering
			\includegraphics[width=0.7\linewidth]{img/cycle-requirements}
			\caption{l'ingegneria dei requisiti è un processo iterativo.}
			\label{fig:cycle-requirements}
		\end{figure}

		 Fare riferimento alla figura \ref{fig:cycle-requirements}. Durante la prima fase, si studia il \textbf{dominio}:
		 \begin{itemize}
		 	\item il system-as-is (il mondo prima che il software venga implementato).
		 	\begin{itemize}
		 		\item organizzazione aziendale: come è strutturata, le dipendenze, gli obbiettivi strategici, le politiche e i flussi di lavoro, le procedure;
		 		\item il dominio di applicazione: concetti, obbiettivi, compiti, vincoli, regolamentazioni all'interno dell'organizzazione;
		 		\item punti di forza e di debolezza;
		 	\end{itemize}
		 	\item gli stakeholders del sistema:
		 	\begin{itemize}
		 		\item gruppi di individui o singole persone coinvolte dal system-to-be, che potrebbero influenzare lo sviluppo e l'accettazione del sistema;
		 		\item decision makers, manager, esperti di dominio, utenti, clienti, subappaltatori, analisti, sviluppatori.
		 	\end{itemize}
		 \end{itemize}
		 Questo processo produce una forma iniziale di glossario della terminologia da utilizzare.

		 Successivamente inizia la parte di \textbf{elicitazione dei requisiti}\footnote{Elicitazione: è un termine che indica l'induzione di una risposta o di un fenomeno in una situazione o in un contesto specifico. L'elicitazione dei requisiti è la pratica di ricercare e scoprire i requisiti di un sistema da parte di utenti, clienti e altre parti interessate.} (secondo quadrante della figura \ref{fig:cycle-requirements}) nella quale si esplora il problema da risolvere con il software, analizzando il system-as-is, le specificità, le opportunità tecnologiche a disposizione.

		 I requisiti sono accordati con gli stakeholders dopo una fase di negoziazione e durante la quale si identificano i punti di disaccordo, i rischi, le opzioni alternative. Poi si prioritarizzano i requisiti.

		 Segue la fase in cui i requisiti vengono documentati (terzo quadrante nella figura \ref{fig:cycle-requirements})con il \textbf{documento dei requisiti}(RD). La documentazione ha l'obbiettivo di fissare gli scopi del sistema software, le proprietà rilevanti di dominio, le specifiche software e di sistema, le assunzioni e le responsabilità. Il documento deve essere \textbf{strutturato} e \textbf{comprensibile} da tutti gli stakeholders.

		 Nella fase successiva, il documento deve essere convalidato per verificare la presenza di inconsistenze, omissioni, o errori in generale.

		 Il documento rappresenta un contratto. Fornisce una stima dei costi, funziona da preventivo.

		 L'approccio appena descritto nella figura \ref{fig:cycle-requirements} è \textbf{iterativo}.
		 \subsubsection{Obbiettivi del processo RE}
		Gli obbiettivi del processo di ingegneria dei requisiti sono:
		\begin{itemize}

			\item completezza;
			\item consistenza;
			\item adeguatezza;
			\item univocità;
			\item misurabilità;
			\item pertinenza;
			\item fattibilità;
			\item comprensibilità;
			\item buona struttura;
			\item modificabilità;
			\item tracciabilità;
		\end{itemize}
		degli elementi del documento dei requisiti.
		\subsubsection{Errori nel processo RE}
		Gli errori nel documento dei requisiti possono essere (gli errori critici sono in rosso):
		\begin{itemize}
			{\color{red}\item omissione;
			\item contraddizione;
			\item inadeguatezza;
			\item ambiguità;}
			\item inmisurabilità;
			\item rumore;
			\item irrealizzabilità;
			\item strutturazione povera, riferimenti immaturi
		\end{itemize}
			\subsection{Errori}
		Esempi applicati degli errori in rosso.
		\begin{longtable}{|m{0.29\linewidth}|m{0.63\linewidth}|}
			\hline
			\textbf{Errore} & \textbf{Descrizione}\\
			\hline
			\textbf{Omissione} & Un problema del mondo reale non viene riportato in nessun elemento del documento dei requisiti:
			\begin{verse}
				 nessun requisito riguardo lo stato delle porte del treno in caso di una fermata di emergenza.
			\end{verse}\\
			\hline
			\textbf{Contraddizione} & Il documento dei requisiti potrebbe contenere frasi che sono in opposizione, cioè si contraddicono:  \begin{verse}
				\textit{"Le porte devono restare chiuse tra due piattaforme."}
			\end{verse}
			\begin{verse}
				\textcolor{red}{ e }
			\end{verse}
			\begin{verse}
				\textit{"Le porte devono rimanere aperte in caso di una fermata di emergenza."}
			\end{verse}
			sono in contraddizione.\\
			\hline
			\textbf{Inadeguatezza} & Il requisito non è formulato opportunamente per descrivere una caratteristica del mondo reale. Ad esempio, l'affermazione:
			\begin{verse}
				\textit{"I pannelli informativi all'interno del treno dovrebbero mostrare tutti i voli in partenza presso il terminal al quale il treno sta arrivando."}
			\end{verse}
			è inadeguato perché non specifica quali aerei (tutti gli aerei in partenza, gli aerei della prossima ora, quelli in ritardo, solo quelli in orario? Devo mostrare il gate? La porta d'imbarco?). Sono gli stakeholders a dover decidere cosa deve essere mostrato e non è compito degli sviluppatori decidere cosa mostrare.\\
			\hline
			\textbf{Ambiguità} & Gli elementi del RD non dovrebbero lasciare un'interpretazione libera dei requisiti al lettore.
			\begin{verse}
				\textit{"Le porte dovrebbero rimanere aperte non appena il treno si ferma alla piattaforma."}
			\end{verse}\\
			\hline
			\textbf{Inmisurabilità} & È complicato enunciare una caratteristica del mondo reale in modo tale da precludere il confronto tra le opzioni o la sperimentazione di soluzioni. Ad esempio
			\begin{verse}
				\textit{"I pannelli all'interno del treno dovrebbero essere user-friendly."}
			\end{verse}
			non è una caratteristica misurabile.
			\\
			\hline
		\end{longtable}
		\subsection{Vizi}
		I \textbf{vizi} sono generalmente meno gravi rispetto agli errori.
		\begin{longtable}{|m{0.27\linewidth}|m{0.63\linewidth}|}
			\hline
			\textbf{Vizio} & \textbf{Descrizione}\\
			\hline
			\textbf{Rumore} & Elementi del RD che non danno alcuna informazione rispetto alle funzionalità del mondo reale. Ad esempio:
			\begin{verse}
				\textit{"I segnali di divieto di fumare devono essere posti sulle finestre del treno.}
			\end{verse}
			Il rumore è un'informazione irrilevante.
			\\
			\hline
			\textbf{Sovra-specifica} & È una frase del RD che si riferisce al software anziché alla risoluzione di una caratteristica del mondo reale. Ad esempio:
			\begin{verse}
				\textit{"Il metodo setAlarm dovrebbe essere invocato o recepito come un messaggio di allarme."}
			\end{verse}
			\\
			\hline
			\textbf{Irrealizzabilità} & Gli elementi del RD non sono realizzabili. Ad esempio:
			\begin{verse}
				\textit{"I pannelli all'interno del treno dovrebbero mostrare tutti i voli in ritardo alla prossima fermata."}
			\end{verse}
			\\
			\hline
			\textbf{Incomprensibilità}& Gli elementi del RD sono incomprensibili a coloro che ne necessitano. Ad esempio:
			\begin{verse}
				\textit{"Una dichiarazione dei requisiti contiene 5 acronimi"}
			\end{verse}\\
			\hline
			\textbf{Struttura di scarsa qualità} & Gli elementi del RD non sono organizzati in accordo con qualsiasi regola di strutturazione sensata e visibile. Ad esempio:
			\begin{quotation}
				\textit{"Presenza di un intreccio tra il controllo dell'accelerazione e il tracciamento dei treni."}
			\end{quotation}
			\\
			\hline
			\textbf{Referenza in avanti} & Il RD contiene riferimenti a caratteristiche del mondo reale non ancora definite. Ad esempio:
			\begin{verse}
				\textit{"L'uso multiplo del concetto di distanza di arresto nel caso peggiore, prima che la definizione appaia diverse pagine dopo nel RD."}
			\end{verse}\\
			\hline
			\textbf{Rimorso} & Il RD contiene forward reference (l'ingegnere dei requisiti usa termini non ancora definiti). L'ingegnere dei requisiti prova rimorso perché sa che sta facendo un lavoro in modo sbagliato e inizia a scrivere le definizioni tra parentesi.\\
			\hline
			\textbf{Scarsa modificabilità} & Il RD contiene degli elementi che non si propagano automaticamente. Ad esempio:
			\begin{verse}
				\textit{"Utilizzo di valori numerici costanti per quantità soggette a cambiamento."}
			\end{verse}
			\\
			\hline
		\end{longtable}
		Il processo di raccolta dei requisiti cambia in base al tipo di processo software adottato. Ci sono varie tipologie di software:
		\begin{itemize}
			\item \textbf{green-field} vs \textbf{brown-field}: green-field è un termine che si riferisce ad un progetto software che viene scritto da zero, e quindi lascia la massima libertà a chi lo progetta; brown-field è usato per identificare i sistemi legacy e di conseguenza vincola il progettista al contesto esistente.
			\item \textbf{customer-driven} vs \textbf{market-driven}: il primo si riferisce ad un progetto che è guidato dalle scelte del cliente, mentre il secondo dipende dalle scelte del mercato;
			\item \textbf{in-house} vs \textbf{outsourced}: sviluppo di un software all'interno di un'organizzazione aziendale; sviluppo di un prodotto software a partire dalla raccolta dei requisiti e concluso solitamente con la stipulazione di un contratto.
			\item \textbf{single-product} vs \textbf{product-line}: il primo identifica lo sviluppo di una singola linea di prodotti (es. lo sviluppo di un treno); il secondo una linea di prodotti con differenti modelli (es. smartphone).
		\end{itemize}
		\begin{figure}[th]
			\centering
			\includegraphics[width=0.7\linewidth]{img/grafico-torta-problema-requisiti-1955}
			\caption{Standish report del 1955: 8000 progetti di 350 aziende americane.}
			\label{fig:grafico-torta-problema-requisiti-1955}
		\end{figure}
		Il problema di gestire i requisiti è notevole se si guarda il grafico che riporta i risultati di un report storico condotto nel 1955 e che ha coinvolto 8000 progetti provenienti da 350 aziende diverse (figura \ref{fig:grafico-torta-problema-requisiti-1955}).
		 I principali motivi di fallimento dei progetti software dello Standish report del 1955 sono riportati nel grafico in figura \ref{fig:grafico-torta-motivi-fallimento}.
		 \begin{figure}[bh]
		 	\centering
		 	\includegraphics[width=0.7\linewidth]{img/grafico-torta-motivi-fallimento}
		 	\caption{motivi di fallimento dei progetti nello Standish report del 1955.}
		 	\label{fig:grafico-torta-motivi-fallimento}
		 \end{figure}
		 I problemi legati ai requisiti rimangono costanti. Gli errori legati ai requisiti sono tra quelli più numerosi, più costanti e più costosi. Un esempio di errore di una proprietà di dominio definita inadeguatamente è quello del sistema di frenaggio dell'A320 riportato nella figura \ref{fig:a320-braking-logic}.
		 \begin{figure}[th]
		 	\centering
		 	\begin{eqnarray*}
		 			\begin{tblr}{m{0.2\linewidth}m{0.645\linewidth}}
		 			SofReq\colon & reverse=\mlq on \mrq \iff WheelPulses=\mlq on\mrq\\
		 			\color{red}ASM \colon & \color{red} reverse= \mlq on\mrq  \iff ReverseThrustEnabled \\ &
		 			\color{red} WheelPulses= \mlq on \mrq \iff WheelsTurning\\
		 			\colorbox{yellow}{$Dom\colon$}  & \colorbox{red}{	\sout{$MovingOnRunway \iff WheelsTurning$}}\\
		 			\hline[dashed]
		 			SysReq\colon & ReverseThrustEnabled \iff MovingOnRunway
		 		\end{tblr}
		 	\end{eqnarray*}
		 	\caption{l'errore che ha causato l'incidente aereo a Varsavia}
		 \label{fig:a320-braking-logic}
		 \end{figure}

		Il sistema frenante è composto da:

		\begin{enumerate}
			\item  Spoiler a terra.

			\begin{itemize}
				\item Se selezionato "ON", gli spoiler a terra si estenderanno se vengono soddisfatte le seguenti condizioni "a terra":

				\item entrambi i montanti oleo (ammortizzatori) sono compressi su entrambi i carrelli di atterraggio principali (il carico minimo per comprimere un ammortizzatore è di 6300 kg),

				\item oppure la velocità delle ruote è superiore a 72 nodi su entrambi i carrelli di atterraggio principali.
			\end{itemize}

			\item  Invertitori di motore.

			\begin{itemize}
				\item Se selezionato "ON", gli invertitori del motore si attiveranno se viene soddisfatta la seguente condizione "a terra": gli ammortizzatori sono compressi su entrambi i carrelli di atterraggio principali.
			\end{itemize}

			\item  Freni delle ruote.
			\begin{itemize}

				\item Le condizioni sopra menzionate (velocità delle ruote superiore a 72 kts ed entrambi gli ammortizzatori compressi) non vengono utilizzate per attivare i freni.

				\item Con la modalità primaria del sistema frenante, i freni possono essere utilizzati non appena la velocità delle ruote su entrambi i carrelli di atterraggio è superiore a 0,8 $V_0$ dove $V_0$ è una velocità di riferimento calcolata dalla BCSU.

				\item Con la modalità alternativa del sistema frenante, i freni possono essere utilizzati non appena l'interruttore A/SKID-NOSE WHEEL STEERING è stato selezionato in posizione OFF dall'equipaggio.
			\end{itemize}
		\end{enumerate}
	L'aquaplaning che ha interessato il D-AIPN della Lufthansa durante il rollout si è manifestato in varie fasi del rollout, probabilmente fin dall'inizio.

	Potrebbe essere causato contemporaneamente da:

	\begin{itemize}
		\item strato d'acqua irregolare fino a diversi millimetri che ricopre la superficie della pista;
		\item  elevata velocità di atterraggio;
		\item  notevole usura di tre o quattro pneumatici del carrello di atterraggio principale.
	\end{itemize}

	Sulla base dell'analisi delle tracce sulla superficie della pista e della natura dei danni ai pneumatici si è dovuto supporre che nella parte finale della pista ricoperta di cemento l'aquaplaning si presentasse nella forma più sviluppata come scivolamento di ruote bloccate sul cuscino di vapore acqueo. Ciò ha causato la radicale diminuzione del coefficiente di attrito, ciò che è stato confermato dalla registrazione dell'avanzamento della frenata trovata nel registratore dei dati di volo.

	Questo argomento è stato elaborato dalla società "Consulting Lotniczy AVIAPOL", Poznan.

	Dunque ciò che ha inibito l'attivazione dei sistemi di frenaggio sono stati principalmente:
	\begin{enumerate}
		\item il contatto di una sola ruota sulla pista;
		\item il fenomeno dell'acquaplanning.
	\end{enumerate}
	che sono riconducibili ad una definizione errata delle proprietà di dominio.

	Gli \textbf{ostacoli} all'introduzione delle pratiche corrette all'ingegneria dei requisiti:
	\begin{itemize}
		\item raccogliere i requisiti è costoso in termini di tempo e non è possibile sapere in anticipo se il contratto verrà stipulato;
		\item le scadenze;
		\item economicità dei requisiti: potrebbero esserci pochi studi ed è difficile stimare il costo dei requisiti;
		\item il progresso nei processi RE è più difficile da misurare rispetto ai processi di implementazione e progettazione;
		\item i documenti dei requisiti potrebbero diventare grandi, complessi e non aggiornati; potrebbe essere troppo distante dal prodotto finale e i RD potrebbero essere implementati in maniera non strutturata.
	\end{itemize}
	\subsection{RE nei progetti agili}
	Nei progetti \textbf{agili}, il processo di RE è gestito utilizzando un processo ciclico a spirale, in cui il primo ciclo è molto veloce e in cui viene abbozzato il progetto e che rappresenta una base per i cicli successivi. Forti presupposti affinché l'agilità abbia successo:
	\begin{enumerate}
		\item  tutti i ruoli degli stakeholder sono riducibili ad un unico ruolo;
		\item progetto sufficientemente piccolo da poter essere assegnato a un team unico, piccolo e con un'unica sede (programmatori/tester/manutentori);
		\item l'“Utente” può interagire tempestivamente ed efficacemente;
		\item la funzionalità può essere fornita in modo rapido, coerente e incrementale dall'essenziale al
		meno importante (non è richiesta alcuna priorità);
		\item  aspetti non funzionali, ipotesi ambientali, obiettivi, opzioni alternative, i rischi possono ricevere poca attenzione;
		\item poca documentazione richiesta per il coordinamento del lavoro e la manutenzione del prodotto;
		precisione dei requisiti non richiesta; la verifica prima della codifica è meno importante del rilascio anticipato;
		\item è improbabile che le modifiche ai requisiti richiedano un importante refactoring del codice.
	\end{enumerate}
	\section{Comprensione del dominio e elicitazione dei requisiti}
	Rappresenta la prima fase della spirale della figura \ref{fig:cycle-requirements}.
	Per acquisire informazioni riguardo i requisiti vengono eseguite certe attività:
	\begin{itemize}
		\item studiare il system-as-is:
		\begin{itemize}
			\item organizzazione aziendale: la struttura, le dipendenze, gli obbiettivi strategico, le politiche, i flussi di lavoro, le procedure operazionali;
			\item il dominio di applicazione: i concetti, gli obbiettivi, i compiti, i vincoli, i regolamenti;
			\item analisi di problemi con il system-as-is: le cause, le conseguenze;
		\end{itemize}
		\item analizzare le opportunità tecnologiche, le condizioni di mercato più recenti;
		\item identificare gli \textbf{stakeholders} di sistema;
		\item identificare gli obbiettivi da migliorare, i vincoli organizzativi e tecnologici del system-to-be; le opzioni alternative per soddisfare gli obbiettivi, per assegnare responsabilità; gli scenari di un ipotetica interazione con l'ambiente software, i requisiti sul software, le assunzioni sull'ambiente.
	\end{itemize}
	 \subsection{Tecniche di elicitazione}
	 Vi sono due tecniche principali di elicitazione:
	 \begin{enumerate}
	 	\item \textbf{Artefact-driven}: orientata agli artefatti disponibili
	 	\item \textbf{Stakeholder-driven}: orientata al coinvolgimento degli stakeholders.
	 \end{enumerate}
	 \subsection{Analisi degli stakeholders}
	 La cooperazione degli stakeholder è essenziale per un RE di successo. Deve essere selezionato un campione rappresentativo per assicurare una copertura adeguata, comprensiva del problema del mondo reale.
	 La selezione avviene in base alla posizione dell'organizzazione, al ruolo nel prendere le decisioni, il tipo di contributo alla conoscenza che porta, livello di esposizione ai problemi, interessi personali, potenziali conflitti (differenti tipologie di utenti possono avere necessità distinte e anche in conflitto perché magari vogliono guadagnare una qualche forma di posizione di vantaggio all'interno dell'azienda).

	 Acquisire la conoscenza dagli stakeholder è difficile perché:
	 \begin{itemize}
	 	\item sorgenti di informazioni diverse, punti di vista in contrapposizione;
	 	\item difficoltà ad accedere alle persone chiave (più in alto sono nella gerarchia aziendale, più probabile è la loro indisponibilità perché coprono ruoli critici);
	 	\item background, terminologia e cultura differenti (che vuol dire che, persone diverse potrebbero utilizzare una terminologia differente);
	 	\item conoscenza tacita, necessità nascoste: alcuni soggetti potrebbero sottintendere concetti perché sono dati per scontato;
	 	\item dettagli irrilevanti: l'utente esprime, attraverso esempi, dettagli irrilevanti;
	 	\item resistenza al cambiamento, politiche interne, competizione: le persone cambiano difficilmente le proprie abitudini (dal ramo operativo, al ramo dirigenziale);
	 	\item cambio del personale, cambiamenti nell'organizzazione, cambiamenti nelle priorità.
	 \end{itemize}
	 L'ingegnere dei requisiti deve avere diverse soft-skills: abilità di comunicazione, capacità di ascolto, stabilire una relazione di fiducia, capacità di riformulare le conoscenze e di ristrutturarle in continuazione.

	 \subsubsection{Background study} Serve per capire il dominio in cui l'azienda opera. Consiste nel raccogliere conoscenza di base a partire da informazioni provenienti dall'organizzazione aziendale che si sta studiando. I documenti devono contenere informazioni riguardo:
	 \begin{itemize}
	 	\item l'\textbf{organizzazione:} avviene attraverso la consultazione di grafici organizzativi (ad esempio un albero in cui vengono rappresentati i ruoli dei dipendenti), piani di business (quali sono le strategie messe in atto dall'azienda per raggiungere gli obbiettivi), report finanziari, verbali delle riunioni (cosa è stato detto durante le riunioni);
	 	\item \textbf{dominio:} attraverso libri, articoli, sondaggi, regolamentazioni, report su sistemi simili che operano nello stesso dominio;
	 	\item \textbf{system-as-is:} consultando la documentazione delle procedure aziendali, regole di business, documenti scambiati, le richieste di cambiamento, i report che contengono dei difetti riscontrati, delle lamentele degli utenti (in alcuni ambiti, è obbligatorio per legge);
	 \end{itemize}
	 L'ingegnere dei requisiti utilizza lo studio di background per comunicare più efficacemente con gli stakeholders.

	 L'attività di raccolta dei requisiti potrebbe essere onerosa in termini di tempo, perché il materiale da analizzare durante il background study potrebbe essere datato, contenere dettagli irrilevanti. Una possibile soluzione è data dall'utilizzo di metadati per filtrare la documentazione (intesa come fonte di informazioni provenienti dall'organizzazione aziendale).

	 \subsubsection{Data collection}
	 Consiste nella raccolta di dati, fatti aziendali, quindi informazioni quantitative. Questo tipo di valutazioni serve per ottenere requisiti \textbf{non funzionali}. Un problema nell'elicitazione dei requisiti. è ottenere dati affidabili e riuscire ad interpretarli correttamente.
	 \subsubsection{Questionari} Consiste nel somministrare agli stakeholders, una lista di domande (aperte o chiuse) per raccogliere informazioni che riguardano il system-as-is. Le domande possono essere a crocette, a risposta multipla, espresse in una scala ordinale. Questa tecnica è efficace per raccogliere informazioni velocemente e in maniera economica, senza la necessità di recarsi presso gli stakeholders.

	 La formulazione delle domande all'interno dei questionari richiede di prestare molta attenzione,   in particolare a non suggerire le risposte agli stakeholders o inserire un bias nelle domande. Ogni domanda deve essere chiara, non ambigua. Ci sono delle linee guida che possono essere seguite per formulare correttamente i questionari. È importante fare un \textbf{cross-check}, riformulando più volte la stessa domanda per assicurare la non presenza di inconsistenze nelle risposte.
	 I questionari potrebbero essere controllati da terze parti per verificare la consistenza delle risposte.
\end{document}