\begin{figure}[ht]
	\centering
		%\hrule\bigskip % show \columnwidth
	
	\noindent
	\begin{tikzpicture}
		\node[draw] at (0,0) {%
			\parbox{\dimexpr\columnwidth-(\pgfkeysvalueof{/pgf/inner xsep})*2\relax}{%
				\begin{enumerate}[nosep, wide]
					\item L'organizzatore decide di pianificare un meeting attraverso il sistema di gestione dei meeting che gli chiede di scegliere quando organizzare un meeting, selezionando in un intervallo temporale, e la lista dei partecipanti.
					\item Il sistema verifica che l'organizzatore abbia i permessi per avanzare la richiesta (valutando il suo ruolo) e in caso affermativo, conferma all'organizzatore che l'attività di scheduling è stata iniziata.
					\item Il sistema chiede a tutti i partecipanti nella lista di inviare i vincoli in termini di date e il luogo dove fare il meeting.
					\item Quando un ciascun partecipante invia i vincoli, il sistema li verifica e se sono validi, li registra e notifica ciascun partecipante.
					\item Il sistema determina la data e la locazione del meeting.
					\item Il sistema software notifica i partecipanti e l'organizzatore riguardo la data e il luogo del meeting.
				\end{enumerate}%
		}};
	\end{tikzpicture}
	\label{fig:esempio-scenario}
	\caption{Un esempio di scenario di un sistema di pianificazione dei meeting:}
\end{figure}